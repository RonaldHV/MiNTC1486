\documentclass{MiNTC1486}

\usepackage[T1]{fontenc}
\usepackage[utf8]{inputenc}
\usepackage{graphicx}
\usepackage[hidelinks]{hyperref}
\usepackage{lipsum}

% %-- Datos del trabajo de grado --
\Titulo{ESTE ES EL TÍTULO DEL TRABAJO DE GRADO O PROYECTO, que lo puede escribir en minúsculas}
\AutorA{NOMBRE DEL PRIMER AUTOR}
\AutorB{Nombre del segundo autor}
\AutorC{}
\Institucion{Universidad con nombre}
\Facultad{Facultad de Ciencias Ocultas}
\Departamento{Departamento de objetos perdidos}
\Programa{Programa-Dor}
\Ciudad{CIUDAD}
\Anho{2022}
\Mes{Febrero}
\Dia{29}
\Tipotrabajo{Trabajo de grado para el titulo de Jedi,\\ u otro tipo de trabajo}
\Director{Director:\\Nombre de mi Sensei}
\JuradoA{Sisabencomosoy Paraquémeinvitan\\ Cargo con experiencia}
\JuradoB{Jurado 2\\ Profesor de facultad}


\begin{document}

% -- Cubierta (Opcional) --
\Cubierta

% --Portada --
\Portada

% --Aceptación (Opcional) --
\Aceptacion

% -- Dedicatoria (Opcional) --
\Dedicatoria{Este trabajo se lo dedico hasta el señor de los domicilios}
\Dedicatoria[Nombre del dedicador o dedicadores]{Esto va para todos}

% -- Agradecimentos(Opcional) --
\newpage
\thispagestyle{empty}
\begin{center}
    AGRADECIMIENTOS
\end{center}
\Agradecientos[Quien da las gracias]{%
Gracias totales

\lipsum[1]
}
%
\Agradecientos[Bienagradecido]{%
\lipsum[3]
}

% -- Contenido --
\Contenido

% -- Lista de figuras --
\Listadefiguras

% % -- Lista de tablas --
\Listadetablas

% -- Resumen --
\Resumen{\lipsum[1]}

% -- Introducción --
% Esta es la introducción del documento
\chapter*{INTRODUCCIÓN}
\addcontentsline{toc}{chapter}{INTRODUCCIÓN}
%
\lipsum[2-4]


% -- Capítulo 1 --
% Este es el primer capítulo del documento
\chapter{ESTE ES UN TÍTULO DE PRIMER NIVEL}


\lipsum[1-2]

\section{Este es un título de segundo nivel}
%
\lipsum[3-4]

\subsection{Este es un título de tercer nivel.}
%
\lipsum[5]

\subsubsection{Este es un título de cuarto nivel.}
%
\lipsum[7]

\subsection{Este es otro  título de tercer nivel.}
%
\lipsum[7]

%
% -- Capítulo 2 --
% Este es el segundo capítulo del documento
\chapter{EL CAPÍTULO 2 DEL DOCUMENTO}
%
\lipsum[1]

Esta es la manera de presentar las figuras. La Figura \ref{figuraNorma}, muestra el ejemplo.

\begin{figure}[ht]
    \vspace{\baselineskip}
    \caption{Esta es la forma de incluir las figuras, con un \emph{caption} extenso para notar cómo se muestra en la figura}
    \label{figuraNorma}
    \centering
    \includegraphics[width=0.7\textwidth]{Cap_2/Fig_Cap_2/Norma}
    \vspace{\baselineskip}
\end{figure}

La Figura \ref{figuraPortada} es un archivo con extensión .pdf, que muestra la primera página de la NTC~1486~\cite{NTC1486}.

\begin{figure}[ht]
    \vspace{\baselineskip}
    \caption{Esta es una figura en formato .pdf}
    \label{figuraPortada}
    \centering
    \includegraphics[width=0.5\textwidth]{Cap_2/Fig_Cap_2/Portada}
    \vspace{\baselineskip}
\end{figure}

\section{Tablas}
La Tabla \ref{refdelatabla}, muestra un ejemplo de cómo quedan las tablas.

\begin{table}[!ht]
    \vspace{\baselineskip}
    \caption{Esta es la descripción de la tabla}
    \label{refdelatabla}
    \centering
    \begin{tabular}{cccc}
        \hline
        \hline
        Col 1   & Col 2 & Col 3 &  Col 4\\
        \hline
        fila 1  & 1     & 2     & 3\\
        fila 2  & 4     & 5     & 6\\
        fila 3  & 7     & 8     & 9\\
        \hline
        \hline
    \end{tabular}
\end{table}



\section{Este sería un subtítulo}
%
\lipsum[3-4]

%
% -- CONCLUSIONES --
% Estas son las onclusiones del documento
\chapter{CONCLUSIONES}
%
\lipsum[1-4]

%
% -- BIBLIOGRAFÍA --
% \chapter*{BIBLIOGRAFÍA}
\addcontentsline{toc}{chapter}{\normalsize BIBLIOGRAFÍA}

\bibliographystyle{IEEEtran}
\bibliography{Bibliografia/misreferencias}


\end{document}
