% Este es el segundo capítulo del documento
\chapter{EL CAPÍTULO 2 DEL DOCUMENTO}
%
\lipsum[1]

Esta es la manera de presentar las figuras. La Figura \ref{figuraNorma}, muestra el ejemplo.

\begin{figure}[ht]
    \vspace{\baselineskip}
    \caption{Esta es la forma de incluir las figuras, con un \emph{caption} extenso para notar cómo se muestra en la figura}
    \label{figuraNorma}
    \centering
    \includegraphics[width=0.7\textwidth]{Cap_2/Fig_Cap_2/Norma}
    \vspace{\baselineskip}
\end{figure}

La Figura \ref{figuraPortada} es un archivo con extensión .pdf, que muestra la primera página de la NTC~1486~\cite{NTC1486}.

\begin{figure}[ht]
    \vspace{\baselineskip}
    \caption{Esta es una figura en formato .pdf}
    \label{figuraPortada}
    \centering
    \includegraphics[width=0.5\textwidth]{Cap_2/Fig_Cap_2/Portada}
    \vspace{\baselineskip}
\end{figure}

\section{Tablas}
La Tabla \ref{refdelatabla}, muestra un ejemplo de cómo quedan las tablas.

\begin{table}[!ht]
    \vspace{\baselineskip}
    \caption{Esta es la descripción de la tabla}
    \label{refdelatabla}
    \centering
    \begin{tabular}{cccc}
        \hline
        \hline
        Col 1   & Col 2 & Col 3 &  Col 4\\
        \hline
        fila 1  & 1     & 2     & 3\\
        fila 2  & 4     & 5     & 6\\
        fila 3  & 7     & 8     & 9\\
        \hline
        \hline
    \end{tabular}
\end{table}



\section{Este sería un subtítulo}
%
\lipsum[3-4]
